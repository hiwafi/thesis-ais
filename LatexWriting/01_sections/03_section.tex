\section{Research Methodology} \label{method}

In this chapter the methodology used to develop the model will be discussed. 
The discussion on different parameters in the vessel's journey data will be discussed here. 
This includes the mining and merging of the features. The method used to develop the ship's speed model will be discussed in this chapter. 
This consists of the parameter used to develop the model. Ultimately, the model is then used to predict the ship's fuel consumption.

\subsection{Data Preprocessing}

\begin{itemize}
    \item Two data sources are imported. {\tt AIS\_weather\_H\_ok2\_copy.csv} \\ and {\tt AIS\_weather\_h\_rename\_copy.csv}. The information from the latter comma delimited 
    file will be used for calculating the ship Speed Through Water (STW).  
    The information required is the true north current direction. Which is obtained from the vector component of the Northward and Southward current.
    \item This dataframe will be merged with the main dataframe from \\ the file {\tt AIS\_weather\_H\_ok2\_copy.csv}.
    \item Omission of the journey data between Ronne and Sassnitz
    \item SOG threshold is applied to omit ship mooring and maneuvering to accurately represent the ship's steady state operation 
    \cite{Abebe.2020,BalBesikci.2016,Gkerekos.2019,Yang.2020}. This threshold is selected as 5 knots according to \cite{Abebe.2020}
    \item The AIS data from June is filtered. This data will be used as validation data to check the model's performance.
\end{itemize}
 
\subsection{Data Analysis}
\begin{itemize}
    \item The features are represented in a histogram plot. For the feature Current speed, anomaly is detected. Certain spike is detected around \numrange[range-phrase = --]{0.01}{0.03} \verb|m/s|. Reasons unknown. The data is retained, including the spike, until a definitive answer can be found.
    \item OPEN QUESTION : What is the necessity of feature standardization / normalization ? Normalization is required for ANN as model training requires the value between 0 and 1. But in case of RFR, there is no such requirement. Through testing, data standardization also does not seem to improve the model's performance. 
\end{itemize}

\begin{sidewaysfigure}
    \includegraphics[width=\linewidth,height=\textheight,keepaspectratio]{02_figures/outputhist.png}
    \caption{Histogram of the features}
    \label{sidewaysfig:hist1}
\end{sidewaysfigure}

\newpage

\begin{itemize}
    \item The correlation of the features against SOG are determined. It is found that :
    \begin{itemize}
        \item Draught
        \item Course Over Ground (COG)
        \item heading
        \item Wind Speed
        \item Current Speed
        \item True Current direction
    \end{itemize}
    Have relatively stronger correlation to SOG comapred to other features, albeit the correlation is a weak one
    \item The correlation between the features is displayed using the following the heat map. From the heat map it can be observed that between these features:
    \begin{itemize}
        \item Waveheight and wind wave swell height
        \item Waveheight and wind wave height
        \item Windwaveswellheight and wave period
    \end{itemize}
    Have a strong correlation between each other.  
    \item Open topic: 
    \begin{itemize}
        \item Feature reduction is possible, \cite{Abebe.2020} suggested high feature correlation filter, the filter suggest that two features which has a high correlation $(>90\%)$ is to be combined into a single feature. But the author is unsure whether this combination is physically sensible. Hence, this filter is yet to be applied for feature reduction. 
        \item Some of these features can be connected through wave equations, but the author has not found an equation which could relate these features.
    \end{itemize}
    \item The random forest regressor could not function when \verb|NaN| values are present. With that, the missing values are filled in using the {\tt imputer} function. The missing values are filled in by means of \verb|KNN|.
\end{itemize}

\newpage

\begin{figure}
    \includegraphics[width=\linewidth,height=\textheight,keepaspectratio]{02_figures/heatmap_corr.png}
    \caption{Correlation Heat Map}
    \label{fig:heatmap1}
\end{figure}

\subsection{Modelling}

\begin{itemize}
    \item The data is split into 80:20 ratio. But considering the validation data, it is split into approximately 73:18:9.
    \item The model is then trained using Random Forest Regression (RFR). Additional training is also performed using Decision Tree Regressor (DTR). DTR model performance will be used as a benchmark as it is also a tree-based modelling method with similar methodology to RFR.
    \item The computational time of DTR is significantly faster than RFR
    Model Evaluation    
\end{itemize}

\subsection{Predicting Ship's Speed Through Water STW}

\begin{itemize}
    \item The ship STW can be calculated using vector component of the SOG and current speed. The direction used will be according to True North. \cite{Yang.2020,Zhou.2020}
    \item 
\end{itemize}
